\chapter{Tests et Validation}
  
  \subsection*{Test 01: Choix de l'établissement}
    \begin{enumerate}
    \item Démarrer l'application pour la première fois, un menu vous sera proposé pour effectuer le choix de l'établissement~[Fig.~\ref{fig:01}].
    \item Sélectionnez les établissements auxquels vous voulez vous abonner (on peut annuler pour quitter l'application).
    \end{enumerate}

    \underline{Résultat attendu}: Vous serez redirigé vers le menu principal de l'application.

  \subsection*{Test 02: Maintien du choix des établissements}
    \begin{enumerate}
    \item Quittez l'application, et redémarrez-la.
    \end{enumerate}

    \underline{Résultat attendu}: L'application aura sauvegardé vos abonnements, et vous redirigera sur la page d'accueil de l'application.

  \subsection*{Test 03: Evénements}
    \begin{enumerate}
    \item Une fois les abonnements configurés, sélectionner le menu des événements dans le menu principal~[Fig.~\ref{fig:02}].
    \end{enumerate}

    \underline{Résultats attendus}:
    \begin{itemize}
    \item Si l'application est connectée à Internet: Les événements de tous les établissements auxquels l'application est abonnée sont affichés~[Fig.~\ref{fig:04}].
    \item Sinon: Les événements ayant déjà été téléchargés sont affichés.
    \end{itemize}

  \subsection*{Test 04: Validité des informations extraites d'un évenement}
    \begin{enumerate}
    \item Depuis l'accueil, appuyer sur Evénements.
    \item Naviguer dans la liste des événements.
    \item Appuyer sur un événement afin d'afficher la vue détaillée~[Fig.~\ref{fig:06}].
    \item Appuyer sur le bouton \emph{Plus}, ou bien \emph{Menu->Ajouter}.
    \end{enumerate}

    \underline{Résultat attendu}: Redirection vers la page de création d'événements de Google Calendar, avec les informations correspondantes à celles contenues dans l'événement.

  \subsection*{Test 05: Ajout d'un événement au smartphone}
    \begin{enumerate}
    \item Effectuer le Test 04.
    \item Valider l'ajout de l'événement.
    \end{enumerate}

    \underline{Résultat attendu}: L'événement sera ajouté à l'agenda du téléphone.


  \subsection*{Test 06: Annuaire}
    \begin{enumerate}
    \item Sélectionner \emph{Annuaire} dans le menu d'accueil.
    \end{enumerate}

    \underline{Résultats attendus}:
    \begin{itemize}
    \item Si l'application est connectée à Internet: on arrive sur l'écran de la~figure~\ref{fig:07}.
    \item Sinon : un toast nous prévient qu'on est déconnecté.
    \end{itemize}

  \subsection*{Test 07: Rechercher dans l'Annuaire}
    \begin{enumerate}
    \item Accéder à l'annuaire à partir du menu principal.
    \item Effectuer une recherche~[Fig.~\ref{fig:08}], cela affichera tous les contacts qui correspondent à la recherche, parmis les établissements auxquels l'application est abonnée.
    \end{enumerate}

    \underline{Résultat attendu}: Les résultats se réduisent à ceux qui correspondent à la recherche.

  \subsection*{Test 08: Validité des informations extraites}
    \begin{enumerate}
    \item Depuis l'accueil appuyer sur \emph{Annuaire}
    \item Effectuer une recherche de contact.
    \item Effectuer une touche longue sur un contact.
    \item Sélectionner \emph{Ajouter un nouveau contact}.
    \end{enumerate}

    \underline{Résultat attendu}: Redirection vers la page de création de contact, pré-remplie avec les informations correspondantes à celles contenues dans le résultat de recherche.

  \subsection*{Test 09: Ajout d'un contact au téléphone}
    \begin{enumerate}
    \item Effectuer le Test 08.
    \item Valider l'ajout/l'édition du contact
    \end{enumerate}

    \underline{Résultat attendu}: Les informations se trouvent dans les contacts du téléphone.

  \subsection*{Test 10: Appeler un contact}
    \begin{enumerate}
    \item Depuis le menu d'accueil, appuyer sur \emph{Annuaire}.
    \item Sélectionner un contact possédant un numéro de téléphone.
    \item Effectuer une touche longue sur un contact.
    \item Sélectionner \emph{Appeler}
    \end{enumerate}

    \underline{Résultat attendu}: L'application démarrera un appel vers ce contact.

  \subsection*{Test 11: Envoyer un mail à un contact}
    \begin{enumerate}
    \item Depuis le menu d'accueil appuyer sur \emph{Annuaire}.
    \item Sélectionner un contact possédant une adresse mail.
    \item Effectuer une touche longue sur un contact.
    \item Sélectionner \emph{Envoyer un email}
    \end{enumerate}

    \underline{Résultat attendu}: L'application ouvre le gestionnaire d'email du smartphone avec le destinataire rempli avec l'adresse email du contact.

  \subsection*{Test 12: Exporter un emploi du temps}
    \begin{enumerate}
    \item S'abonner à un établissement permettant l'exportation d'emplois du temps (ex: Bordeaux1).
    \item Depuis le menu d'accueil, appuyer sur \emph{Export emploi du temps}.
    \item Sélectionner un niveau d'études~[Fig.~\ref{fig:09}].
    \item Sélectionner une filière et un groupe~[Fig.~\ref{fig:10}].
    \item Appuyer sur \emph{Exporter}
    \end{enumerate}

    \underline{Résultat attendu}:
    \begin{itemize}
    \item Si l'application est connectée à Internet: l'emploi du temps choisi est intégré au calendrier du smartphone.
    \item Sinon: Le test échoue dès la première étape, un toast apparait pour indiquer qu'une connexion est nécessaire.
    \end{itemize}

  \subsection*{Test 13: Afficher le plan du campus de Bordeaux1}
    \begin{enumerate}
    \item S'abonner à un établissement permettant ayant un plan de campus (ex: Bordeaux1).
    \item Sélectionner \emph{Plan du Campus} dans le menu d'accueil.
    \end{enumerate}

    \underline{Résultat attendu}:
    \begin{itemize}
    \item Si le GPS est activé: ouvre l'application Google Maps centrée sur le campus de Bordeaux1.
    \item Sinon: Affiche une image du plan du campus de Bordeaux1.
    \end{itemize}
