\chapter{Analyse des besoins}

Nous allons analyser dans ce chapitre les différents besoins fonctionnels et non-fonctionnels de notre application.\\
Nous allons donc utiliser un système de points pour évaluer la difficulté des tâches. On se basera sur une séquence de Fibonacci arrondie (1, 2, 3, 5, 8, 13, 20, 40, 60, 100), avec 100 reflétant une tâche d'une grande difficulté.\\
Les priorités des besoins sont classées sur cette échelle: optionnel, faible, moyen, important, vital.
\section{Besoins fonctionnels}
\subsection{Fonctionnalités principales}
\subsubsection{Choix des établissements}
Etant donné que l'application présentera des services en commun à plusieurs établissements, l'utilisateur doit pouvoir s'abonner aux établissements de son choix, s'il souhaite accéder à leurs propres informations [Fig.~\ref{fig:01}]. \\

\begin{itemize}
\renewcommand{\labelitemi}{$\bullet$}
\item Priorité: vital
\item Difficulté: 1/100
\item Risques: mise à jour des liens si les adresses des établissements sont modifiées.
\item Validation: s'assurer que les sélections et les modifications des abonnements s'effectuent correctement.
\end{itemize}

\subsubsection{Gestion des actualités}
En accédant au menu des actualités, l'utilisateur disposera d'une interface à travers laquelle il recevra le flux des actualités des établissements auxquels il est abonné. Par défaut il aura alors la possibilité de voir les événements propres aux établissements auxquels il s'est abonné. Il pourra à tout moment changer de catégorie grâce au menu des catégories [Fig.~\ref{fig:04}] depuis la même fenêtre. De plus, il peut choisir entre voir la totalité des actualités des établissements auxquels il s'est souscrit, ou voir uniquement les actualités propres aux établissements qu'il a sélectionné dans les filtres.  [Fig.~\ref{fig:03}]. L'utilisateur doit pouvoir lire l'annonce de ces événements [Fig.~\ref{fig:04}], et doit pouvoir facilement rajouter cet événement au calendrier de son smartphone (Google Calendar). Cette action redirigera l'utilisateur vers le calendrier, dans lequel l'événement sera créé avec les informations récoltées. L'utilisateur pourra donc modifier, valider ou annuler la procédure. \\


\begin{itemize}
\renewcommand{\labelitemi}{$\bullet$}
\item Priorité: vital
\item Difficulté: 40/100
\item Difficultés techniques: extraction des informations pertinentes.
\item Risques: changement de structure du contenu à parser (flux RSS et pages HTML).
\item Validation: assurer l'intégrité des informations extraites.
\end{itemize}

\subsubsection{Accès à l'annuaire}
L'utilisateur doit pouvoir effectuer facilement les tâches suivantes:\\
\begin{itemize}
\renewcommand{\labelitemi}{$\bullet$}
\item Rechercher un contact dans l'annuaire par son nom et/ou prénom.
\item Visualiser les informations d'un contact.
\item Rajouter un contact à la liste des contacts de l'utilisateur (créer un nouveau contact ou éditer existant).
\item Envoyer un mail au contact.
\item Appeler la personne (si un numéro de téléphone est disponible).
\item Visiter le site web de la personne (à condition qu'il en existe un).
\end{itemize}

\wl La recherche de personnes dans l'annuaire se fera en fonction des établissements auxquels l'utilisateur s'est abonné. On pourra donc choisir grâce à un filtre entre rechercher une personne dans l'annuaire propre à un ou plusieurs établissements, ou la rechercher dans tous les annuaires.
Lors de l'ajout d'un contact, l'utilisateur sera redirigé vers la page de création d'un nouveau contact, dans laquelle les champs pertinents seront pré-remplis, en fonction des informations collectées. L'utilisateur pourra donc modifier, valider ou annuler la procédure. \\

\begin{itemize}
\renewcommand{\labelitemi}{$\bullet$}
\item Priorité: vital
\item Difficulté: 40/100
\item Difficultés techniques:
\begin{itemize}
\item Avec LDAP: mise en place du protocole.
\item Sans LDAP: parsage des informations.
\end{itemize} 
\item Risques: changement de structure du contenu à parser.
\item Validation: assurer l'intégrité des informations extraites.
\end{itemize}

\subsubsection{Mises à jour}

Notre application pouvant fonctionner hors connexion, les mises à jour de flux RSS étant de type pull, nous avons initialement pensé qu'il serait important de proposer les mises à jour des flux que nous parsons. Par exemple après une longue période sans connexion, nous voulions sauter sur l'occasion pour proposer des mises à jour à l'utilisateur. Or la solution que nous avons adoptée consiste à mettre à disposition un bouton \emph{refresh} pour que l'utilisateur puisse effectuer une demande de mise à jour, non seulement cette option est préférable pour les personnes ayant un forfait Internet limité, ou encore pour l'utilisation à l'étranger, mais nous avons estimé que cette fonctionnalité ne serait pas très utile sachant que les établissements en question ne proposent que rarement des mises à jour(quelques unes par semaine). Cela ne justifie donc pas le besoin d'avoir un système de mises à jour automatiques qui au lieu de faciliter l'expérience utilisateur, pourrait juste avoir des impacts sur les performances globales. En effet, faire des mises à jour trop souvent et de manière inutile fait perdre du temps à l'utilisateur et demande plus de travail de la part du CPU et mène donc à une consommation excessive de la batterie du téléphone.
L'application devra garder en mémoire les derniers éléments téléchargés. Ainsi lorsque l'on n'a pas accès à Internet, on pourra accéder à ceux-ci. De plus, lors des mises à jour, on téléchargera uniquement les nouveaux événements. Ceci permettra à l'application d'économiser du temps d'exécution. \\

\begin{itemize}
\renewcommand{\labelitemi}{$\bullet$}
\item Priorité: important
\item Difficulté: 40/100
\item Risques: mauvaise gestion de la mémoire.
\item Difficultés techniques: éviter de télécharger de nouveau des informations déjà collectées.
\item Validation: vérifier l'intégration des nouvelles informations.
\end{itemize}


\subsection{Fonctionnalités secondaires}
Dans un deuxième temps, nous comptons implémenter quelques options supplémentaires, qui ne rentrent pas dans le cadre du projet initial, mais qu'on estime être assez pratiques et utiles, et permettront de compléter l'application.

\subsubsection{Abonnements et filtrage d'informations}
Après avoir choisi lors du premier démarrage de l'application, les établissements auprès desquels l'utilisateur souhaite s'abonner~[Fig.~\ref{fig:01}], il pourra à tout moment revenir sur son choix à travers le menu des préférences.
Il pourra également en fonction de ses abonnements actifs visualiser la totalité des informations propres à un, plusieurs ou tous les établissements grâce à un outil de filtrage. Cet outil se présente sous la forme d'un menu quasiment identique à celui du choix des abonnements et sera accessible uniquement depuis l'écran d'accueil. \\
D'autres filtres concernant les événements seront proposés afin de permettre à l'utilisateur de maintenir une liste d'événements favoris marqués à l'aide d'une étoile qu'il pourra modifier à tout moment. Il pourra également choisir de visualiser uniquement les événements à venir, ou encore les événements non lus. \\

\begin{itemize}
\renewcommand{\labelitemi}{$\bullet$}
\item Priorité: optionnel
\item Difficulté: 60/100
\item Difficultés techniques: s'assurer que l'abonnement simultané à un grand nombre d'établissements ait un impact mineur sur la stabilité de l'application.
\item Risques: saturation trop rapide de la mémoire cache.
\item Validation: affichage cohérent des informations liées aux abonnements actifs en un temps raisonnable (de l'ordre de 3 à 4 secondes par établissement).
\end{itemize}

\subsubsection{Emploi du temps}
L'utilisateur étudiant doit pouvoir récupérer son emploi de temps du semestre et l'intégrer à Google Calendar. Il devra simplement sélectionner dans la liste des emplois du temps, celui qui lui correspond. \\
Deux autres choix lui seront également proposés: il pourra visualiser l'emploi du temps directement en ligne dans un navigateur internet, ou alors télécharger une version au format\textit{PDF} pour le stocker sur son téléphone.

\begin{itemize}
\renewcommand{\labelitemi}{$\bullet$}
\item Priorité: optionnel
\item Difficulté: 60/100
\item Difficultés techniques: extraction des informations en fonction du choix de l'utilisateur.
\item Risques: changement du format de l'emploi du temps.
\item Validation: assurer l'intégrité des données extraites.
\end{itemize}

\subsubsection{Plan du campus}
L'utilisateur pourra naviguer sur le plan du campus en utilisant Google Maps. Des pointeurs vers les bâtiments principaux du campus ainsi que vers les points de restauration à proximité seront affichés et pourront être filtrés. \\
Un outil de recherche sera également mis à disposition de l'utilisateur et lui permettra de rapidement trouver un bâtiment, un snack, ou un service proposé aux étudiants. \\
Le service de localisation intégré aidera également l'utilisateur à se repérer sur le campus en mettant fréquemment à jour sa position actuelle à l'aide des satellites GPS.

\begin{itemize}
\renewcommand{\labelitemi}{$\bullet$}
\item Priorité: optionnel
\item Difficulté: 60/100
\item Difficultés techniques: maîtriser l'API de Google Maps.
\end{itemize}



\section{Besoins non fonctionnels}
\subsection{Besoins de performance}
L'application CampusUB1 est destinée à toute personne étudiant ou travaillant sur le campus de l'université. Elle devra donc être capable de tourner sur les derniers smartphones, tout en restant compatible avec des smartphones plus anciens et donc moins puissants puisque tout étudiant n'a pas forcément les moyens d'aborder un smartphone récent ayant une forte puissance de calcul et une grande quantité de RAM. CampusUB1 ne sera donc pas gourmande en ressources et pourra exécuter ces services de manière assez rapide même sur les smartphones moins performants.\\
Nous testerons l'application sur les appareils suivants: LG Optimus 2X (P990), Sony Xperia S, Samsung Galaxy S3.\\

\begin{itemize}
\renewcommand{\labelitemi}{$\bullet$}
\item Priorité: faible
\item Difficulté: 40/100
\item Difficultés techniques: bonne gestion de l'espace mémoire et de l'actualisation.
\item Risques: implémention complexe et potentiellement longue.
\item Validation: vérification que l'utilisation reste assez fluide et ne consomme pas beaucoup de ressources (à préciser un seuil de consommation de RAM).
\end{itemize}  

\subsection{Besoins de fiabilité}
Etant donné que les différents services mis à disposition par CampusUB1 à ses clients s'avèrent souvent être d'un besoin professionnel, il est essentiel que les données affichées soient cohérentes et en accord avec les données d'origine. Si un chercheur du LaBRI recherche depuis notre application le numéro de la salle où a lieu sa conférence, il a tout intérêt à récupérer une information correcte. CampusUB1 ne peut tout de même pas garantir l'exactitude de toutes les informations données et ne se tient donc pas responsable d'éventuelles erreurs, sachant que l'application se base sur des algorithmes de reconnaissance de textes qui peuvent eux-mêmes être erronés d'origine. Il est évident que certaines questions de sécurité courantes se posent sachant que l'application nécessite une connexion à Internet continue, mais vu que CampusUB1 utilise uniquement des données publiques dans les services qu'elle fournit, la question de confidentialité des données ne se pose pas. \\

\begin{itemize}
\renewcommand{\labelitemi}{$\bullet$}
\item Priorité: moyen
\item Difficulté: 60/100
\item Difficultés techniques: fiabilité de l'algorithme d'analyse textuelle pour détecter les dates des événements.
\item Risques: création d'un événement avec des données erronées.
\item Validation: simulation de l'algorithme d'analyse textuelle sur suffisamment de données pour avoir un taux de réussite acceptable (à déterminer).
\end{itemize}  

\subsection{Besoins d'intégrité de données}
CampusUB1 doit accéder à certaines ressources personnelles du client pour pouvoir fonctionner, ce qui pose un problème de protection des données. Le calendrier et le répertoire du client sont les ressources auxquelles l'application est susceptible d'accéder, et on devra s'assurer que toute perte éventuelle de données déjà existantes ne soit pas engendrée par une erreur provenant de l'application. L'erreur humaine étant évidemment toujours possible. Toute modification éventuelle de données privées de l'utilisateur se fera donc impérativement avec son avis. \\

\begin{itemize}
\renewcommand{\labelitemi}{$\bullet$}
\item Priorité: moyen
\item Difficulté: 40/100
\item Risques: perte des informations existantes sur le smartphone.
\item Validation: monitoring du comportement de l'application.
\end{itemize} 


\subsection{Besoins organisationnels}
Etant donné que nous travaillerons sur la plateforme Android, le langage Java nous est imposé vu que l'API Android est fournie en Java.
L'IDE dans lequel nous avons choisi d'implémenter notre application est Eclipse pour de nombreuses questions pratiques. Les développeurs d'Android disposent d'un plugin officiel pour Eclipse, qui facilite le travail du développeur en lui fournissant une panoplie d'outils tels que les différentes perspectives dont celle de débogage qui nous sera très utile. Ce plugin comporte également un émulateur d'appareils Android, ce qui nous permettra de facilement tester l'application sur une multitude d'appareils virtuels. De plus, l'API Android est très documentée et est maintenue fréquemment.