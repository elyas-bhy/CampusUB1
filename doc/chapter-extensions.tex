\chapter{Extensions et améliorations possibles}
Il existe un grand nombre d’améliorations envisageables pour notre application qui est censée évoluer constamment après son déploiement.
Dans un premier temps, il faudrait ajouter le support pour d’autres éventuels établissements appartenant au campus tels que l’IUT de Bordeaux 1 ou l’INRIA. 
Il est alors à prévoir qu’on risque très probablement d’avoir d’autres pages HTML organisées différemment à parser, et donc de nouveaux modules à intégrer.\\\\
 
Lorsqu’on a commencé à implémenter l’annuaire, notre choix initial était d’utiliser le serveur LDAP de l’Université de Bordeaux 1 ainsi que celui du LaBRI pour accéder aux contacts des établissements en question. Le problème étant que le serveur LDAP du LaBRI n’est pas accessible depuis l’extérieur sans authentification, il ne nous restait pas d’autre choix que de parser la page HTML du LaBRI en question. Une éventuelle amélioration possible aurait alors été d’utiliser le serveur LDAP sécurisé et de demander à l’utilisateur de s’authentifier s'il veut effectuer une recherche de contacts au sein du LaBRI. Cela aurait nettement amélioré le temps de recherche dans l’annuaire. Mais dans le cas pratique, l’utilisateur perdrait énormément de temps à entrer ses identifiants afin de lancer la recherche et accéderait finalement encore moins vite aux contacts qu’avec notre implémentation actuelle. Cette amélioration est donc à voir uniquement d’un point de vue d’optimisation des  performances de l’application, et non du côté facilité d’utilisation.\\\\

Pour le plan du campus, on pourrait également donner plus de détails concernant les marqueurs en donnant plus d’informations concernant celui-ci. Par exemple pour les bâtiments, on pourrait proposer à l’utilisateur de consulter le plan intérieur du bâtiment en question, ou pour les restaurants d’afficher les horaires d’ouverture et de fermeture ainsi que leur numéro de téléphone.\\
Une autre extension possible serait de pouvoir calculer les itinéraires routiers à pied ou en voiture afin de faciliter l’accès au campus, et à ses différents bâtiments. Il serait alors possible d’indiquer à l’utilisateur le chemin à effectuer pour se rendre à pied d’un bâtiment du campus à un autre.\\
Nous avons également réfléchi à une amélioration de l’outil de recherche intégré, qui serait capable de nous proposer dynamiquement une liste de suggestions correspondant à notre recherche en proposant un service d'auto-complétion.\\\\

En ce qui concerne l’emploi du temps, une fonctionnalité qui aurait pu accélérer l’affichage de l’emploi du temps ainsi que de l’améliorer esthétiquement aurait été de créer nous-même une vue de l’emploi du temps au sein de l’application en récupérant les informations nécessaires, et donc ne pas avoir à rediriger l’utilisateur vers le navigateur internet qui prend du temps à s’ouvrir ainsi qu’à afficher les emplois du temps en format HTML.\\