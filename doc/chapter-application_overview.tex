\chapter{L'Application CampusUB1}
\section{Aperçu de l'application}

\subsection{Actualités}
Une des fonctionnalités principales de l'application est la récupération, l'affichage et l'enregistrement des événements dans le calendrier de l'utilisateur. 
Pour cela, nous prévoyons de récupérer les flux RSS les plus utiles de l'université Bordeaux 1\cite{fluxBDX1}, notamment les flux disponibles sur la page d'accueil du site de l'université (dont les actualités).
En ce qui concerne le site du LaBRI, il n'y a aucun flux RSS et il faudra donc parser directement du code HTML.

\subsection{Annuaire}
Il sera également possible de consulter les annuaires des établissements concernés, afin d'en extraire les informations désirées et les ajouter aux contacts du smartphone.
Afin d'implémenter l'annuaire au sein de notre application Android, il est évident qu'il faut exploiter les annuaires de Bordeaux 1~\cite{AnnuaireBdx1} et du LaBRI. En ce qui concerne le LaBRI, le serveur LDAP n'est pas accessible depuis l'extérieur (sans authentification).Nous devrons donc intégrer l'annuaire à travers des requêtes GET, dans le but de parser les résultats obtenus. Pour le moment ceci semble être la meilleure solution envisageable. L'accès sans authentification au serveur LDAP de l'université de Bordeaux1 nous est permis, ce qui facilitera grandement l'intégration de la recherche dans l'annuaire pour cet établissement, ainsi que la maintenabilité du logiciel.


\subsection{Utilitaires}
\subsubsection{Plan du campus}
Dans un deuxième temps, nous prévoyons de mettre en place un plan du campus, potentiellement intégré à l'application Maps du smartphone.
Ce service permettrait donc à l'utilisateur d'avoir un aperçu plus détaillé du campus en affichant les différents bâtiments et leur noms.
On poura éventuellement aussi afficher les emplacements des différents services dédiés aux étudiants (cafétéria, restaurant universitaire, bibliothèque universitaire, ...).

\subsubsection{Emploi du temps}
Un autre service qui pourrait être très utile en particulier pour les étudiants de Bordeaux1 est l'accès et la sauvegarde locale de leur emplois du temps en fonction de leurs filières, des groupes, et des différentes options suivies. 
Pour cela, nous parserons les fichiers XML des emplois du temps utilisant la plateforme Celcat\cite{EdTxml}, afin de mettre à disposition une copie locale de l'emploi du temps dans l'agenda du smartphone.