\chapter{Architecture}
\section{Aperçu de l'architecture}

\begin{figure}[h]
  \label{fig:architecture}
  \center
  \setlength\fboxsep{5pt}
  \setlength\fboxrule{0.5pt}
  \fbox{\includegraphics[width=0.9\textwidth]{resources/architecture.png}}
\end{figure}

L'application est organisée en plusieurs packages, chacun définissant des services différents. On distingue les modules les plus importants: Settings, Directory, et Events.

\newpage
\subsection{Settings}

\begin{figure}[h!]
  \label{fig:preferences_mod}
  \center
  \setlength\fboxsep{5pt}
  \setlength\fboxrule{0.5pt}
  \fbox{\includegraphics[width=0.9\textwidth]{resources/preferences_mod.png}}
\end{figure}

Ce module s'occupe de la gestion des préférences de l'utilisateur. Toutes les opérations de lecture/écriture des préférences sont effectuées à travers ce module, notamment la gestion des abonnements et des filtres.

\newpage
\subsection{Directory}

\begin{figure}[h!]
  \label{fig:contacts_mod}
  \center
  \setlength\fboxsep{5pt}
  \setlength\fboxrule{0.5pt}
  \fbox{\includegraphics[width=0.9\textwidth]{resources/contacts_mod.png}}
\end{figure}

Ce module s'occupe de la gestion de l'annuaire. Il est responsable des services suivants:
\begin{itemize}
\renewcommand{\labelitemi}{$\bullet$}
\item Parsage des réponses LDAP / pages HTML.
\item Stockage et sauvegarde des données.
\item Affichage graphique des contacts de l'annuaire.
\end{itemize}

\newpage
\subsection{Events}

\begin{figure}[h!]
  \label{fig:events_mod}
  \center
  \setlength\fboxsep{5pt}
  \setlength\fboxrule{0.5pt}
  \fbox{\includegraphics[width=1.0\textwidth]{resources/events_mod.png}}
\end{figure}

Ce module s'occupe de la gestion des événements. Il est responsable des services suivants:
\begin{itemize}
\renewcommand{\labelitemi}{$\bullet$}
\item Parsage des flux RSS / pages HTML.
\item Stockage et sauvegarde des données.
\item Affichage graphique des événements.
\end{itemize}
