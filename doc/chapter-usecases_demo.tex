\chapter{Exemple d'utilisation}

Les images suivantes représentent un premier aperçu de l'interface graphique de l'application. Nous imaginons donc un scénario d'utilisation qui montre le déroulement d'un usage typique.

\begin{figure}
  \begin{minipage}[t]{8cm}
    \centering
    \includegraphics[width=0.8\textwidth]{resources/ui_preview/01}
    \caption{Lors du premier lancement de l'application, l'utilisateur doit sélectionner ses établissements. Une fois ses choix effectués, il sont enregistrés et ne seront plus demandés. Les établissements sont modifiables via les préférences.}
    \label{fig:01}
  \end{minipage}
  \hspace{+20pt}
  \begin{minipage}[t]{8cm}
    \centering
    \includegraphics[width=0.8\textwidth]{resources/ui_preview/02}
    \caption{Vue d'accueil de l'application}
    \label{fig:02}
  \end{minipage}
  \hspace{-60pt}
\end{figure}


\begin{figure}
  \begin{minipage}[t]{8cm}
    \centering
    \includegraphics[width=0.8\textwidth]{resources/ui_preview/03}
    \caption{Vue des filtres des résultats en fonction des établissements préalablement choisis. Cette fonctionnalité est accessible en touchant le logo (situé en haut à gauche) de l'application uniquement depuis l'écran d'accueil.}
    \label{fig:03}
  \end{minipage}
  \hspace{+20pt}
  \begin{minipage}[t]{8cm}
    \centering
    \includegraphics[width=0.8\textwidth]{resources/ui_preview/04}
    \caption{Vue des événements. Pour lire la news et arriver sur l'écran de la figure~\ref{fig:06}, il suffit d'appuyer sur la news correspondante. Par défaut, toutes les news sont affichées. On a également la possibilité de les afficher par catégories. La liste des catégories est disponible via la touche menu.}
    \label{fig:04}
  \end{minipage}
  \hspace{-60pt}
\end{figure}


\begin{figure}
  \begin{minipage}[t]{8cm}
    \centering
    \includegraphics[width=0.8\textwidth]{resources/ui_preview/05}
    \caption{Liste des catégories pour afficher les news.}
    \label{fig:05}
  \end{minipage}
  \hspace{+20pt}
  \begin{minipage}[t]{8cm}
    \centering
    \includegraphics[width=0.8\textwidth]{resources/ui_preview/06}
    \caption{Vue détaillée d'une news. Le bouton ``plus'' situé en haut à droite permet d'ajouter l’événement au calendrier du smartphone.}
    \label{fig:06}
  \end{minipage}
  \hspace{-60pt}
\end{figure}


\begin{figure}
  \begin{minipage}[t]{8cm}
    \centering
    \includegraphics[width=0.8\textwidth]{resources/ui_preview/07}
    \caption{Vue de l'annuaire.}
    \label{fig:07}
  \end{minipage}
  \hspace{+20pt}
  \begin{minipage}[t]{8cm}
    \centering
    \includegraphics[width=0.8\textwidth]{resources/ui_preview/08}
    \caption{Vue des résultats de l'annuaire.}
    \label{fig:08}
  \end{minipage}
  \hspace{-60pt}
\end{figure}


\begin{figure}
  \begin{minipage}[t]{8cm}
    \centering
    \includegraphics[width=0.8\textwidth]{resources/ui_preview/09}
    \caption{Export de l'emploi du temps. L'application demande la promotion seulement la première fois et il faudra aller dans les paramètres pour la modifier ultérieurement.}
    \label{fig:09}
  \end{minipage}
  \hspace{+20pt}
  \begin{minipage}[t]{8cm}
    \centering
    \includegraphics[width=0.8\textwidth]{resources/ui_preview/10}
    \caption{Sélection des groupes pour l'export de l'emploi du temps.}
    \label{fig:10}
  \end{minipage}
  \hspace{-60pt}
\end{figure}